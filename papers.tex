\documentclass[12pt]{article}
\usepackage[utf8]{inputenc} % allow utf-8 input
\usepackage[T1]{fontenc}    % use 8-bit T1 fonts
\usepackage{hyperref}       % hyperlinks
\usepackage{url}            % simple URL typesetting
\usepackage{booktabs}       % professional-quality tables
\usepackage{amsfonts}       % blackboard math symbols
\usepackage{nicefrac}       % compact symbols for 1/2, etc.
\usepackage{microtype}      % microtypography
\usepackage{amsmath}                      
\usepackage{mathtools}
\usepackage{graphicx}
\usepackage{float}
\usepackage{subcaption}
\usepackage{wrapfig}
\usepackage{verbatim}
\usepackage{xcolor}
\usepackage{verbatim}
\usepackage{amssymb}
\usepackage{mathtools, nicefrac}

\graphicspath{{figures/}}

\newcommand{\plotfigureS}[2]{
\begin{figure}[H]
\centering
\includegraphics[width=#2\linewidth]{#1}
\caption[#1]{#1}
\label{#1}
\end{figure}
}


\begin{document}



 {\LARGE\noindent\textbf{NIPS 2017}}
\begin{description}
  \item[$\bullet$ Visual interaction networks] Visual interaction networks \cite{watters2017visual}

\end{description}

 {\LARGE\noindent\textbf{ICRA 2019}}
\begin{description}
  \item[$\bullet$ Variational Policy] Vpe: Variational policy embedding for
    transfer reinforcement learning \cite{arnekvist2018vpe}, fig.\ref{vpe}, fig.\ref{vpe2}

\end{description}







 {\LARGE\noindent\textbf{ICLR 2019}}
\begin{description}
  \item[$\bullet$ Unsupervised Physics] Unsupervised Discovery of Parts, Structure, and Dynamics
\cite{xu2019unsupervised}, fig.\ref{unsupervisedDyn1}, fig.\ref{unsupervisedDyn2}
  \item[$\bullet$ Bounce and Learn] Bounce and Learn: Modeling Scene Dynamics with Real-World Bounces
\cite{purushwalkam2019bounce}, fig.\ref{bounceAndLearn1},
fig.\ref{bounceAndLearn2}
  \item[$\bullet$ Particle Dynamics] Learning particle dynamics for manipulating
    rigid bodies, deformable objects, and fluids \cite{li2018learning}, fig.\ref{particleDynamics}, fig.\ref{particleDynamics2}

\end{description}



 {\LARGE\noindent\textbf{CVPR 2018}}
\begin{description}
  \item[$\bullet$ Disentangling components of dynamics] Recognize Actions by
    Disentangling Components of Dynamics \cite{Zhao_2018_CVPR},
    fig.\ref{disentVideo}, fig.\ref{disentVideo2}
  \item[$\bullet$ Soccer VR] Soccer on your tabletop \cite{rematas2018soccer},
    fig.\ref{soccer}
\end{description}



 {\LARGE\noindent\textbf{NIPS 2018}}
\begin{description}
  \item[$\bullet$ DDPAE] Learning to decompose and disentangle representations
    for video prediction \cite{hsieh2018learning}, fig.\ref{ddpae}
\item[$\bullet$ OODP] Object-oriented dynamics predictor
  \cite{kanazawa_2019_cvpr}, fig.\ref{oodp}, fig.\ref{oodp2}
\item[$\bullet$ Beliefs about Dynamics from Behavior] Where do you think you're going?: Inferring beliefs about dynamics from behavior \cite{reddy2018you}, fig.\ref{beliefsdynamics}
\item[$\bullet$ Attractor Dynamics] Learning attractor dynamics for generative
  memory \cite{wu2018learning}, \href{https://neuronaldynamics.epfl.ch/online/Ch17.html}{Attractor dynamics}
  

\item[$\bullet$ Differentiable point clouds] Unsupervised learning of shape and pose with differentiable point clouds \cite{insafutdinov2018unsupervised}, fig.\ref{shapecampose}

\item[$\bullet$ Unsupervised video segmentation for RL] Unsupervised video object segmentation for deep reinforcement learning \cite{goel2018unsupervised}, fig.\ref{videosegRL}


\item[$\bullet$ DL Prob Dynamics] Deep reinforcement learning in a handful of trials using probabilistic dynamics models \cite{chua2018deep}, fig.\ref{pets}
  
\item[$\bullet$ LCP physics \cite{de2018end}]

\item[$\bullet$ Neural ODEs \cite{chen2018neural}]

\item[$\bullet$ Flexible Neural Representation for Physics Prediction \cite{mrowca2018flexible}]


\end{description}

 
 {\LARGE\noindent\textbf{CVPR 2019}}


\begin{description}
  \item[$\bullet$ Sim-to-Real via Sim-to-Sim] Sim-To-Real via Sim-To-Sim: Data-Efficient Robotic Grasping via
Randomized-To-Canonical Adaptation Networks \cite{James_2019_CVPR},
fig.\ref{simtosim}
\item[$\bullet$ 3DHuman] Learning 3d human dynamics from video
  \cite{kanazawa_2019_cvpr}, fig.\ref{3dhuman1}, fig.\ref{3dhuman2}
\item[$\bullet$ MemInMemLSTM] Memory in Memory: A Predictive Neural Network for
  Learning Higher-Order Non-Stationarity From Spatiotemporal Dynamics \cite{Wang_2019_CVPR}, fig.\ref{mimLSTM1}, fig.\ref{mimLSTM2}

\item[$\bullet$ Active Learning] Actively Seeking and Learning From Live Data
  \cite{Teney_2019_CVPR}, fig.\ref{activeLearning}, fig.\ref{activeLearning2}


\end{description}


{\LARGE\noindent\textbf{arXiv}}


\begin{description}
  \item[$\bullet$ Discovering physical concepts] Discovering physical concepts with neural networks \cite{iten2018discovering}
  \item[$\bullet$ Toussaint and Tenenbaum RSS paper] Differentiable Physics and Stable Modes for
Tool-Use and Manipulation Planning \cite{toussaint2018differentiable}, fig.\ref{rsstoussaint}
  \item[$\bullet$ VRNNs for residual physics] Augmenting physical simulators with stochastic neural networks: Case study of planar pushing and bouncing \cite{ajay2018augmenting}
  \item[$\bullet$ Differentiable Soft Objects] ChainQueen: A Real-Time Differentiable Physical Simulator for Soft Robotics \cite{hu2018chainqueen}
  \item[$\bullet$ Rapid Trial-and-Error] The Tools Challenge: Rapid
    Trial-and-Error Learning in Physical Problem Solving \cite{allen2019tools},
    solving tasks similar to the PHYRE Facebook dataset \cite{bakhtin2019phyre}
  \item[$\bullet$ Interpretable intuitive physics model] Interpretable intuitive
    physics model \cite{ye2018interpretable}, fig.\ref{interpphys}, fig.\ref{interpphys2}
  \item[$\bullet$ Unsupervised intuitive physics] Unsupervised intuitive physics
    from visual observations \cite{ehrhardt2018unsupervised},
    fig.\ref{unsupervisedBalls},
    \href{https://geometry.cs.ucl.ac.uk/projects/2018/unsupervised-intuitive-physics/}{Interesting
    dataset}

    \href{https://ths.rwth-aachen.de/research/projects/hypro/benchmarks-of-continuous-and-hybrid-systems/}{Example
      SysID problems, continous, discrete, hybrid, etc.}
    
    \href{https://myrtle.ai/how-to-train-your-resnet-7-batch-norm/}{Batch Norm;
      Why DL does not work for RL}


  \item[$\bullet$ DL from control perspective] Deep Learning Theory Review: An Optimal Control and Dynamical Systems Perspective \cite{liu2019deep}

    
\end{description}



\bibliography{papers}
\bibliographystyle{ieeetr}
\plotfigureS{simtosim}{1.0}
\plotfigureS{3dhuman1}{1.0}
\plotfigureS{3dhuman2}{1.0}
\plotfigureS{mimLSTM1}{1.0}
\plotfigureS{mimLSTM2}{1.0}
\plotfigureS{ddpae}{1.0}
\plotfigureS{oodp}{1.0}
\plotfigureS{oodp2}{1.0}
\plotfigureS{beliefsdynamics}{1.0}
\plotfigureS{pets}{1.0}
\plotfigureS{shapecampose}{1.0}
\plotfigureS{videosegRL}{1.0}
\plotfigureS{disentVideo}{1.0}
\plotfigureS{disentVideo2}{1.0}
\plotfigureS{soccer}{1.0}
\plotfigureS{unsupervisedDyn1}{1.0}
\plotfigureS{unsupervisedDyn2}{1.0}
\plotfigureS{bounceAndLearn1}{1.0}
\plotfigureS{bounceAndLearn2}{1.0}
\plotfigureS{particleDynamics}{1.0}
\plotfigureS{particleDynamics2}{1.0}
\plotfigureS{rsstoussaint}{1.0}
\plotfigureS{vrnnResidual}{1.0}
\plotfigureS{interpphys}{1.0}
\plotfigureS{interpphys2}{1.0}
\plotfigureS{vpe}{1.0}
\plotfigureS{vpe2}{1.0}
\plotfigureS{unsupervisedBalls}{1.0}
\plotfigureS{activeLearning}{1.0}
\plotfigureS{activeLearning2}{1.0}
\end{document}
